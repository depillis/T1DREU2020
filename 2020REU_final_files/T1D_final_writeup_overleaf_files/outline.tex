\documentclass{article}
\usepackage[utf8]{inputenc}

\title{T1D_{final\_writeup}}
\author{Christina Catlett, Daniel Shenker, Rachel Wander, Maya Watanabe}
\date{June 2020}

\begin{document}

\maketitle

\section{Introduction}
\begin{enumerate}
    \item 
    \item Current knowledge
    \begin{itemize}
        \item MCMC
        \item PSO
        \item UKF
        \item MCMC/Bayesian approaches are very common in parameterizing biological systems: here are some of the papers/systems that have used it
        \item Kalman filters are another method, not as widely used (?): here are some examples
    \end{itemize}
    \item Introduce Lotka-Volterra
    \item Introduce T1D
    \begin{itemize}
        \item Why are we using these models?
    \end{itemize}
    \item Parameterization: MCMC v. Kalman filters
    \begin{itemize}
        \item What is the goal of parameterization?
        \item What are the differences?
        \item High-level explanation of each algorithm
        \item When would we use one technique over the other?
        \begin{itemize}
            \item What is required initially? What type of info fits my method?
        \end{itemize}
    \end{itemize}
\end{enumerate}

\section{Problem Set-up}
\begin{enumerate}
    \item Lotka-Volterra
    \begin{itemize}
    \item What are the elements of this model
    \item Equations, general figures etc.
    \item What does parameterization mean in the context of this model? What are the parameters - consistent naming?
    \item Where does this data come from?
\end{itemize}
    \item T1D
    \begin{itemize}
    \item What are the elements of this model
    \item Equations, general figures etc.
    \item What does parameterization mean in the context of this model? What are the parameters - consistent naming?
    \item Where does this data come from?
    \item How did we shift and average data?
    \item Li data only
    \begin{itemize}
        \item For UKF justify single mouse 6 (good), mouse 10 (bad)
    \end{itemize}
\end{itemize}
\end{enumerate}


\section{MCMC}
\subsection{Explanation of theory}
\subsubsection{General tutorial}
\subsection{Parameterize Lotka-Volterra}
\subsection{Parameterize T1D}


\section{Kalman Filters}
\subsection{Explanation of theory}
\subsubsection{General tutorial}
\subsection{Parameterize Lotka-Volterra}
\subsubsection{Joint}
\subsubsection{Dual}
\subsection{Parameterize T1D}
\subsubsection{Joint}
\subsubsection{Dual}

\section{Discussion}
\subsection{General discussion of each algorithm}
This can include discussions about Lotka-Volterra and T1D.
\begin{itemize}
    \item UKFs
    \item DRAM 
    \begin{itemize}
        \item bimodal distributions: some parameters may have more than one local maxima and thus more than one optimal range of values (it will likely depend on interactions with other parameters).
        \item what constitutes "good" chain performance?
        \item averaged vs. mouse 6 data
    \end{itemize}
    \item PSO
\end{itemize}
\subsection{Comparing Results}
\begin{itemize}
    \item Lotka-Volterra comparisons
    \item take mean of 9 post-ukf mice and compare to MCMC means and PSO
    \item mouse 6 comparison MCMC v UKF(joint and dual) v PSO
    \item compare MSE for all (table)
\end{itemize}
\subsection{Improving our algorithms}
\begin{itemize}
    \item Optimal number of parameters MCMC
    \item Running multiple iterations of UKF
\end{itemize}
\subsection{What can we do with both?}
\begin{itemize}
    \item Have noticed when we might want to use one technique over the other?
    \item Combining UKF and MCMC
\end{itemize}
\subsection{Future applications}


\section{References}



\end{document}
